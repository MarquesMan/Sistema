\documentclass{article}
\usepackage[brazil]{babel}
\usepackage[utf8]{inputenc}
\usepackage[bottom=2cm,top=3cm,left=4cm,right=3cm]{geometry}
\usepackage{color}
\usepackage[usenames,dvipsnames,svgnames,table]{xcolor}
\usepackage{hyperref}
\usepackage{amsmath}
\hypersetup{
	colorlinks = true,
	linkcolor = blue
}

\title{Documenta\c{c}\~ao Sistema de Est\'agios}
\author{Gabryel Rigol da Silva \and Lucas Marques Macedo Navarezi \and Victor Ga\'iva}


\begin{document}
\maketitle

\newpage
\tableofcontents
\newpage
\section{Introdu\c{c}\~ao}
O Sistema de Estágios é um site desenvolvido usando o modelo MVC(Model-View-Controller) com a finalidade de agilizar a aprovação da documentação referente aos estágios dos alunos, inicialmente da FACOM-UFMS.\\
Este documento possui a finalidade de demonstrar toda a arquitetura do site e o funcionamento das páginas, para uma boa compreensão das pessoas que irão fazer a manutenção deste sistema.

\section{Configura\c{c}\~ao URL}
Para definirmos qual controlador e qual ação será usado, nós passamos os nomes como parâmetros, assim temos o seguinte formato:\\\\
http://www.sistema.com/index.php?url=controlador/ação/parametro1/parametro2/etc...\\\\
Mas para uma melhor visualiza\c{c}\~ao da url utilizamos o arquivo .htaccess para fazer a reescrita dela, assim ela ficará no seguinte formato:\\\\
http://www.sistema.com/controlador/a\c{c}\~ao/parametro1/parametro2/etc...

\newpage
\section{Arquivos Principais}
\subsection{index.php} \label{sec:index}
O arquivo index.php apenas inclui o arquivo config.php

\subsection{config.php} \label{sec:config}
O arquivo config.php é responsável por definir algumas constantes e incluir o arquivo loader.php;

\subsection{loader.php} \label{sec:loader}
O arquivo loader.php inclui o arquivo global-functions.php, que é responsável por manter todas as funções globais, e também é responsável por instanciar a classe \hyperref[sec:SistemaMVC]{SistemaMVC} que vai controlar todo o início da aplicação.

\subsection{global-functions.php} \label{sec:global-functions}
A função mais importante que temos é a \_autoload, para carregar classes automaticamente.

\newpage
\section{Classes Pad\~ao}
	Dentro do diretório Classes temos os arquivos que contém as classes bases do sistema.\\
	A padronização dos nomes dos arquivos é class-NomeDaClasse.php

	\subsection{class-SistemaMvc.php} \label{sec:SistemaMVC}
		O construtor dessa classe é responsável por ler os parâmetros passados na url, incluir o arquivo que contem o 			controlador, instanciar a classe do controlador e chamar a ação.\\
		Caso o controlador não seja passado, será incluído o controlador Login.
		Caso a ação não seja passada, será chamada a ação index.

	\subsection{class-SistemaBD.php} \label{sec:SistemaBD}
		Classe responsável por fazer a conexão com o banco de dados e realizar consultas, possuímos a seguintes funções:

		\subsubsection{query:}
			Parâmetros: \$stmt(String da consulta SQL), \$data\_array(array dos valores);\\\\
			Funcionamento: Prepara e executa a consulta, caso a consulta não seja feita retorna false, senão retorna o 				resultado da consulta.\\\\
			Exemplo de uso:\\\\
			db-$>$query('SELECT * FROM tabela WHERE campo = ? AND outro\_campo = ?',
			
			\qquad\qquad array( 'valor', 'valor' )) 

		\subsubsection{insert:}
			Parâmetros: \$table, os valores da array são coletadas pela função func\_get\_args();\\\\
			Funcionamento: Cria uma string usando a tabela e a array, e então chama a função query()\\\\
			Exemplo de uso:\\\\
			db-$>$query('tabela', 
	
						// Insere uma linha
						
						array('campo\_tabela' =$>$ 'valor', 'outro\_campo'  =$>$ 'outro\_valor'),
	
						// Insere outra linha
						
						array('campo\_tabela' =$>$ 'valor', 'outro\_campo'  =$>$ 'outro\_valor'),
	
						// Insere outra linha
						
						array('campo\_tabela' =$>$ 'valor', 'outro\_campo'  =$>$ 'outro\_valor')) 

		\newpage
		\subsubsection{update:}
			Parâmetros:\$table (tabela onde está o dado),
			
			\qquad\qquad\$where\_field (coluna de comparação),
			
			\qquad\qquad\$where\_field\_value(valor a ser comparado),
			
			\qquad\qquad\$values(campos a serem alterados e seus valores);\\\\
			Funcionamento: Cria uma string usando a tabela, campo\_where, valor\_where, e a array, e então chama a função query()\\\\
			Exemplo de uso:\\\\
			db-$>$query('tabela', 'campo\_where', 'valor\_where',
	
					\qquad\qquad// Atualiza a linha
					
					\qquad\qquad array('campo\_tabela' =$>$ 'valor', 'outro\_campo'  =$>$ 'outro\_valor')

		\subsubsection{delete:}
			Parâmetros:\$table (tabela onde esta o dado),
			
			\qquad\qquad\$where\_field (campo de comparação),
			
			\qquad\qquad\$where\_field\_value (valor a ser comparado)\\\\
			Funcionamento: Cria uma string usando a tabela, campo\_where, valor\_where e então chama a função query()\\\\
			Exemplo de uso:\\\\
			db-$>$query('tabela', 'campo\_where', 'valor\_where')
			
	\subsection{class-UserLogin.php}
\end{document}